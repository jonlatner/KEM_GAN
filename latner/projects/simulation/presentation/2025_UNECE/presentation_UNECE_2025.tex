% \input{"IAB/latex/TeX-Folienformat.tex"}
\input{"/Users/jonathanlatner/Google Drive/My Drive/IAB/latex/TeX-Folienformat.tex"}

\documentclass[t,8pt,utfx8]{beamer}
\usepackage{booktabs}
\usepackage{setspace}
\usepackage{parskip}
\usepackage{graphicx}
\usepackage{subcaption}
\setbeamertemplate{caption}[numbered]
\newcommand{\sprache}{\englisch}
\renewcommand{\thesubsection}{\alph{subsection})}
\usepackage[cal=pxtx, scr=dutchcal]{mathalpha}
\usepackage{forest}



\definecolor{codegreen}{rgb}{0,0.6,0}
\definecolor{codegray}{rgb}{0.5,0.5,0.5}
\definecolor{codepurple}{rgb}{0.58,0,0.82}
\definecolor{backcolour}{rgb}{0.95,0.95,0.92}


\usepackage{listings}

% Define style for R code
\lstset{
  language=R,
  basicstyle=\ttfamily\small,
  keywordstyle=\color{blue},
  stringstyle=\color{red},
  commentstyle=\color{green},
  showstringspaces=false,
  numbers=left,
  numberstyle=\tiny\color{gray},
  stepnumber=1,
  numbersep=5pt,
  breaklines=true,
  frame=single
}

\newcommand{\btVFill}{\vskip0pt plus 1filll}


\title{Buyer Beware: Understanding the trade-off between utility and risk in CART based models using simulation data}

\subtitle{
UNECE Expert Meeting on Statistical Data Collection 2025, \newline Barcelona, \newline 15-17. October, 2025}

\author{Jonathan Latner, PhD \newline Dr. Marcel Neunhoeffer \newline Prof. Dr. Jörg Drechsler}

\newcounter{noauthorlines}
\setcounter{noauthorlines}{2} % Wert für 2 Autoren über 2 Zeilen. Ggf. anpassen

% %%%%%%%%%%%%%%
% Ende Anpassung
% %%%%%%%%%%%%%%

% \input{"IAB/latex/TeX-Folienformatierung_CD_2019"}
\input{"/Users/jonathanlatner/Google Drive/My Drive/IAB/latex/TeX-Folienformatierung_CD_2019"}

% Modify the section in toc template to enumerate
\setbeamertemplate{section in toc}{%
    \inserttocsectionnumber.~\inserttocsection\par
}

% use for subsections
% \setbeamertemplate{subsection in toc}{}
\setbeamertemplate{subsection in toc}{%
    \setlength{\parskip}{1mm}
        \hskip2mm -- \hskip1mm\inserttocsubsection\par
}


\usepackage{colortbl}
\definecolor{lightgray}{gray}{0.9}

\usepackage{listings} %include R code
\lstdefinestyle{mystyle}{
    backgroundcolor=\color{backcolour},   
    commentstyle=\color{codegreen},
    keywordstyle=\color{magenta},
    numberstyle=\tiny\color{codegray},
    stringstyle=\color{codepurple},
    basicstyle=\ttfamily\tiny,
    breakatwhitespace=false,         
    breaklines=true,                 
    captionpos=b,                    
    keepspaces=true,                 
    numbers=left,                    
    numbersep=5pt,                  
    showspaces=false,                
    showstringspaces=false,
    showtabs=false,                 
    columns=fullflexible,
    frame=single,
    tabsize=2
}
\lstset{style=mystyle}


\begin{document}


\frame[plain]{\titlepage}

\begin{spacing}{1.25}

%%%%%%%%%%%%%%%%%%%%%%%%%%%%%%%%%%%%%%%%
%%%%%%%%%%%%%%%%%%%%%%%%%%%%%%%%%%%%%%%%
\section{Introduction}\label{sec:introduction}
%%%%%%%%%%%%%%%%%%%%%%%%%%%%%%%%%%%%%%%%
%%%%%%%%%%%%%%%%%%%%%%%%%%%%%%%%%%%%%%%%

\begin{frame}[c,plain]
\vskip-4mm
\begin{beamercolorbox}[wd=\boxwidth,ht=22.11mm]{transparent}%
    \vfill%
    \usebeamerfont{title}%
    \leftinsert%
    \MakeUppercase{Section \ref{sec:introduction}: Introduction
} % <- Hier die Überschrift eintragen
\end{beamercolorbox}
\vskip-3mm
\pgfuseimage{rahmenlinie}
\end{frame}

\begin{frame}{Research Question and Approach}
\begin{itemize}
    \item Research Question: 
    \begin{itemize}
        \item  Do common privacy measures accurately capture disclosure risk in synthetic data generated by CART models?
    \end{itemize}
    \item Context:
    \begin{itemize}
        \item Synthetic data increasingly used to share data while preserving privacy.
        \item CART-based SDGs: high statistical utility, relatively low privacy risk.
    \end{itemize}
    \item Privacy Measures Evaluated:
    \begin{itemize}
        \item Identity disclosure risk
        \item Attribute disclosure risk
    \end{itemize}
    \item Data:
    \begin{itemize}
        \item Simulated dataset (Reiter design: 1,000 obs., 4 binary vars., unique case).
        \item Public survey data: Social Diagnosis 2011 (SD2011).
    \end{itemize}
    \item Contribution: 
    \begin{itemize}
        \item Assess validity of empirical disclosure risk measures and their implications for evaluating synthetic data generators.
    \end{itemize}
\end{itemize}
\end{frame}

%%%%%%%%%%%%%%%%%%%%%%%%%%%%%%%%%%%%%%%%
%%%%%%%%%%%%%%%%%%%%%%%%%%%%%%%%%%%%%%%%
\section{The data}\label{sec:data}
%%%%%%%%%%%%%%%%%%%%%%%%%%%%%%%%%%%%%%%%
%%%%%%%%%%%%%%%%%%%%%%%%%%%%%%%%%%%%%%%%

\begin{frame}[c,plain]
\vskip-4mm
\begin{beamercolorbox}[wd=\boxwidth,ht=22.11mm]{transparent}%
    \vfill%
    \usebeamerfont{title}%
    \leftinsert%
    \MakeUppercase{Section \ref{sec:data}: Generate the original and synthetic data
} % <- Hier die Überschrift eintragen
\end{beamercolorbox}
\vskip-3mm
\pgfuseimage{rahmenlinie}
\end{frame}

\begin{frame}[t]\frametitle{Generate original data}
\begin{itemize}
    \item Borrowing from Reiter et al. (2014), we create a data set with $n=1000$ and 4 dichotomous, categorical variables. 
    \item The first 999 observations to be a random sample from a multinomial distribution for all combinations of $var1(0,1), var2(0,1), var3(0,1), var4(0,1)$ except the last one
    \item The last (1000$^{th}$) observation is ($var1=1,var2=1,var3=1,var4=1$). 
\end{itemize}
\end{frame}

\begin{frame}[t]\frametitle{Generate synthetic data}
\begin{itemize}
    \item Synthpop
\end{itemize}
\end{frame}

\begin{frame}[t]\frametitle{Compare original and synthetic data}
\begin{minipage}{0.48\textwidth}
    \begin{figure}
        \centering
        \caption{Frequency}
        \includegraphics[width=\textwidth]{../../graphs/graph_cart_frequency_compare.pdf}
        \label{fig:frequency_compare}
    \end{figure}
\end{minipage}
\hfill
\begin{minipage}{0.48\textwidth}
    \begin{figure}
        \centering
        \caption{Histogram}
        \includegraphics[width=\textwidth]{../../graphs/graph_cart_histogram_compare.pdf}
        \label{fig:histogram_compare}
    \end{figure}
\end{minipage}
\end{frame}

\begin{frame}[t]\frametitle{Compare histogram x 10 synthetic datasets}

\begin{figure}
    \caption{Multiple synthetic data sets does not reduce privacy risk}
    \resizebox{\textwidth}{!}{\includegraphics{../../graphs/graph_cart_histogram_compare_10_v1.pdf}}
    \label{fig:cart_histogram_compare_10}
\end{figure}

\end{frame}

\frame{\frametitle{Summary}
\begin{itemize}
    \item The problem (in our data): Synthetic data from CART models are disclosive
    \item The reason: 
    \begin{itemize}
        \item A record can only be in the synthetic data if it is also in the original data (in this simulated data).   
        \item Or the opposite: if a record is not in the original data, then it can never be in the synthetic data.
    \end{itemize}  
    \item Next section: Can an attacker identify the disclosure?
\end{itemize}
}

%%%%%%%%%%%%%%%%%%%%%%%%%%%%%%%%%%%%%%%%
%%%%%%%%%%%%%%%%%%%%%%%%%%%%%%%%%%%%%%%%
\section{The attack}\label{sec:attack}
%%%%%%%%%%%%%%%%%%%%%%%%%%%%%%%%%%%%%%%%
%%%%%%%%%%%%%%%%%%%%%%%%%%%%%%%%%%%%%%%%
\begin{frame}[c,plain]
\vskip-4mm
\begin{beamercolorbox}[wd=\boxwidth,ht=22.11mm]{transparent}%
    \vfill%
    \usebeamerfont{title}%
    \leftinsert%
    \MakeUppercase{Section \ref{sec:attack}: The attack
} % <- Hier die Überschrift eintragen
\end{beamercolorbox}
\vskip-3mm
\pgfuseimage{rahmenlinie}

\end{frame}

\frame{\frametitle{Describing the attack}
\begin{itemize}
    \item We assume a `strong' attacker similar to the attack model in differential privacy (DP). 
    \item An attacker has the following knowledge
    \begin{itemize}
        \item Knows the SDG model type (i.e. sequential CART).
        \item Knowledge of all observations in the data except the last one.  
        \item The 16 possible combinations that the last one could be.
    \end{itemize}
    \item The attacker sees the synthetic data
    \item The attacker runs the same synthetic data model (SDG) for all of the 16 different possibilities.  
    \item Then they update their beliefs about what the last record could be
\end{itemize}
}

\begin{frame}[t]\frametitle{Histogram of 16 worlds x 10 synthetic datasets}

\begin{figure}
    \caption{}
    \vskip -4mm
    \resizebox{\textwidth}{!}{\includegraphics{../../graphs/graph_attacker_default_v1.pdf}}
    \label{fig:attacker_default}
\end{figure}


\end{frame}



\frame{\frametitle{Summary}
\begin{itemize}
    \item In our attack with our assumptions, the attacker can easily identify the last record
    \item The reason (to repeat): 
    \begin{itemize}
        \item A record can only be in the synthetic data if it is also in the original data (in this simulated data).   
        \item Or the opposite: if a record is not in the original data, then it can never be in the synthetic data.
    \end{itemize}  
    \item Next section: Can we measure this disclosure?
\end{itemize}

}


%%%%%%%%%%%%%%%%%%%%%%%%%%%%%%%%%%%%%%%%
%%%%%%%%%%%%%%%%%%%%%%%%%%%%%%%%%%%%%%%%
\section{Measuring disclosure risk}\label{sec:disclosure}
%%%%%%%%%%%%%%%%%%%%%%%%%%%%%%%%%%%%%%%%
%%%%%%%%%%%%%%%%%%%%%%%%%%%%%%%%%%%%%%%%
\begin{frame}[c,plain]
\vskip-4mm
\begin{beamercolorbox}[wd=\boxwidth,ht=22.11mm]{transparent}%
    \vfill%
    \usebeamerfont{title}%
    \leftinsert%
    \MakeUppercase{Section \ref{sec:disclosure}: Measuring disclosure risk
} % <- Hier die Überschrift eintragen
\end{beamercolorbox}
\vskip-3mm
\pgfuseimage{rahmenlinie}
\end{frame}

\begin{frame}[t]\frametitle{Disclosure risk measures}

\begin{itemize}
    \item The literature on privacy measures for synthetic data is well-developed (Wagner and Eckhoff, 2018).  
    \item Common privacy measures - Synthpop (Raab et al., 2025)
    \begin{itemize}
        \item Identity risk ($repU$): the ability to identify individuals in the data from a set of known characteristics or `keys' ($q$).  
        \item Attribute risk ($DiSCO$): the ability to find out from the keys something, not previously known or `target' ($t$)
    \end{itemize}
    \item Less common measures (Reiter et al., 2014)
    \begin{itemize}
        \item Bayesian risk: the probability that an intruder assigns to the true record after seeing the synthetic data.  
        \begin{itemize}
            \item If this probability is close to the prior, little information is revealed.  
            \item If it is much higher, the intruder has gained information.  
        \end{itemize}
        \item Less common because it is complicated to calculate, especially for more complex real-world data sets as the possible combination of values grows exponentially with the number of columns and possible values per column.
    \end{itemize}
\end{itemize}
\end{frame}

\begin{frame}[t]\frametitle{Results disclosure risk measures}
\begin{minipage}[t]{0.48\textwidth}
    \begin{table}[]
        \centering
        \caption{x 1 synthetic data set (seed = 1237)}
        \resizebox{\textwidth}{!}{% latex table generated in R 4.5.0 by xtable 1.8-4 package
% Wed Aug 13 15:48:17 2025
\begin{tabular}{lrrr}
  \toprule
Data & Identity Risk  & Attribute Risk  & Bayesian Estimate of Risk \\ 
 & ($repU$) & ($DiSCO$) & \\
  \midrule
Original & 0.00 & 0.00 &  1.00 \\ 
  Synthetic & 0.00 & 0.00 & 1.00 \\ 
   \bottomrule
\end{tabular}
}
        \label{table:disclosure_risk_1}
    \end{table}
\end{minipage}%
\hfill%
\begin{minipage}[t]{0.48\textwidth}
    \begin{table}[]
        \centering
        \caption{x 10 synthetic data sets}
        \rowcolors{1}{white}{lightgray}
        \resizebox{\textwidth}{!}{% latex table generated in R 4.5.0 by xtable 1.8-4 package
% Wed Aug 13 15:48:18 2025
\begin{tabular}{lrrrr}
  \toprule
Data & Unique & Identity Risk ($repU$) & Attribute Risk ($DiSCO$) & Bayesian Estimate of Risk \\ 
  \midrule
Original & 1& 0.00 & 0.00 & 1.00 \\ 
  Synthetic 1 &1& 0.00 & 0.00 & 1.00 \\ 
  Synthetic 2 &0& 0.00 & 6.60 & 0.02 \\ 
  Synthetic 3 &1& 0.00 & 0.00 & 1.00 \\ 
  Synthetic 4 &3& 0.00 & 0.00 & 1.00 \\ 
  Synthetic 5 &2& 0.00 & 0.00 & 1.00 \\ 
  Synthetic 6 &1& 0.00 & 0.00 & 1.00 \\ 
  Synthetic 7 &3& 0.00 & 0.00 & 1.00 \\ 
  Synthetic 8 &0& 0.00 & 6.60 & 0.03\\ 
  Synthetic 9 &1& 0.00 & 0.00 & 1.00 \\ 
  Synthetic 10 &1& 0.00 & 0.00 & 1.00 \\ 
  Average & - &  0.00 & 1.32 & - \\ 
   \bottomrule
\end{tabular}
}
        \label{table:disclosure_risk_10}
    \end{table}
\end{minipage}
\end{frame}



\frame{\frametitle{Summary}
\begin{itemize}
    \item Using common privacy measures, CART generates synthetic data with low risk
    \item However (and this is the point):
    \begin{itemize}
         \item We know there is a problem (because we created it)
         \item We know that common measures do not capture the problem 
    \end{itemize} 
    \item Further, $DiSCO$ only captures attribute risk when there is no attribute risk (i.e. no unique observation) 
\end{itemize}
}


%%%%%%%%%%%%%%%%%%%%%%%%%%%%%%%%%%%%%%%%
%%%%%%%%%%%%%%%%%%%%%%%%%%%%%%%%%%%%%%%%
\section{Is this scenario realistic?}\label{sec:reality}
%%%%%%%%%%%%%%%%%%%%%%%%%%%%%%%%%%%%%%%%
%%%%%%%%%%%%%%%%%%%%%%%%%%%%%%%%%%%%%%%%
\begin{frame}[c,plain]
\vskip-4mm
\begin{beamercolorbox}[wd=\boxwidth,ht=22.11mm]{transparent}%
    \vfill%
    \usebeamerfont{title}%
    \leftinsert%
    \MakeUppercase{Section \ref{sec:reality}: Is this scenario realistic?
} % <- Hier die Überschrift eintragen
\end{beamercolorbox}
\vskip-3mm
\pgfuseimage{rahmenlinie}

\end{frame}

\frame{\frametitle{Real world data (SD2011)}

Following the authors of Synthpop (Raab, 2024; Raab et al., 2024), we rely on data from Social Diagnosis 2011 (SD2011).  

In their paper, they generate 5 synthetic data sets to illustrate their method for measuring attribute disclosure by identifying values in the target variable \texttt{depress} from keys: \texttt{sex} \texttt{age} \texttt{region} \texttt{placesize}.  

To illustrate why it is a problem to measure attribute disclosure as the set of records with constant $t$ within $q$, we set $t$ as constant for all observations in all 5 synthetic data sets.  0 was chosen because it is the most frequent value in the variable \texttt{depress} (22\% of all records).  By definition, this reduces attribute disclosure risk.  

In their example, attribute risk is about 9\%. However, when we modify \texttt{depress} so that it is constant (0), the risk \emph{increased} to around 15\%.

Therefore, even though we know risk declined (because we reduced it), $DiSCO$ increases.

}

\begin{frame}[t]\frametitle{Results}

\begin{table}[!h]
    \centering
    \caption{Risk measures for \texttt{depress} from keys: \texttt{sex}, \texttt{age}, \texttt{region}, \texttt{placesize} (SD2011)}
    % \rowcolors{1}{white}{lightgray}
    \input{../../tables/table_disclosure_risk_sd2011_v2.tex}
    \label{tab:attribute_risk_sd2011}
\end{table}



\end{frame}

\begin{frame}[t]\frametitle{Summary}

\begin{itemize}
    \item When we create synthetic data to reduce attribute disclosure risk, $DiSCO$ measure increases

    \item The package authors are aware of the problem 
    \begin{itemize}
        \item  that the $DiSCO$ measure of attribute disclosure risk can indicate a high level of risk for a target variable where a high proportion of records have one level (Raab et al., 2024).
        \item The package includes a flag to allow the user to identify values within a variable that explain most of the disclosures (\texttt{check\_1way}).
    \end{itemize}
    \item We agree, but our example illustrates that the disclosure measure increases, when it should decrease.
    \item The key point is that we show that $DiSCO$ mismeasures risk using real world data
\end{itemize}

\end{frame}


%%%%%%%%%%%%%%%%%%%%%%%%%%%%%%%%%%%%%%%%
%%%%%%%%%%%%%%%%%%%%%%%%%%%%%%%%%%%%%%%%
\section{Conclusion}\label{sec:conclusion}
%%%%%%%%%%%%%%%%%%%%%%%%%%%%%%%%%%%%%%%%
%%%%%%%%%%%%%%%%%%%%%%%%%%%%%%%%%%%%%%%%
\begin{frame}[c,plain]
\vskip-4mm
\begin{beamercolorbox}[wd=\boxwidth,ht=22.11mm]{transparent}%
    \vfill%
    \usebeamerfont{title}%
    \leftinsert%
    \MakeUppercase{Section \ref{sec:conclusion}: Conclusion} % <- Hier die Überschrift eintragen
\end{beamercolorbox}
\vskip-3mm
\pgfuseimage{rahmenlinie}
\end{frame}

\begin{frame}[t]\frametitle{Summary}

\begin{itemize}
    \item Common privacy metrics may fail to detect or even misstate disclosure risk.  
    \item CART-based synthetic data generators reproduce original data with high utility, but offer little protection for disclosive records under default settings.  
    \item Adjusting parameters (adding noise) can reduce disclosure risk, but at the cost of lower utility.  
    \item Key takeaway: users must understand both how SDGs generate data and how privacy measures operate—there is no one-size-fits-all solution.  
\end{itemize}

\end{frame}

\begin{frame}[t]\frametitle{Thank you}

Jonathan Latner: \url{jonathan.latner@iab.de} \\

Reproducible code: \url{https://github.com/jonlatner/KEM\_GAN/tree/main/latner/projects/simulation} 


\end{frame}

\end{spacing}
\end{document}

