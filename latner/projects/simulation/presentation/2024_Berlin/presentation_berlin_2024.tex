% \input{"IAB/latex/TeX-Folienformat.tex"}
\input{"/Users/jonathanlatner/Google Drive/My Drive/IAB/latex/TeX-Folienformat.tex"}

\documentclass[t,8pt,utfx8]{beamer}
\usepackage{booktabs}
\usepackage{setspace}
\usepackage{parskip}
\usepackage{graphicx}
\usepackage{subcaption}
\setbeamertemplate{caption}[numbered]
\newcommand{\sprache}{\englisch}
\renewcommand{\thesubsection}{\alph{subsection})}
\usepackage[cal=pxtx, scr=dutchcal]{mathalpha}



\definecolor{codegreen}{rgb}{0,0.6,0}
\definecolor{codegray}{rgb}{0.5,0.5,0.5}
\definecolor{codepurple}{rgb}{0.58,0,0.82}
\definecolor{backcolour}{rgb}{0.95,0.95,0.92}



\newcommand{\btVFill}{\vskip0pt plus 1filll}


\title{Understanding the trade-off between utility and risk in CART based models using simulation data}
\subtitle{Berlin, \newline 7-8. Oktober, 2024}

\author{Jonathan Latner, PhD \newline Dr. Marcel Neuenhoeffer \newline Prof. Dr. Jörg Drechsler}

\newcounter{noauthorlines}
\setcounter{noauthorlines}{2} % Wert für 2 Autoren über 2 Zeilen. Ggf. anpassen

% %%%%%%%%%%%%%%
% Ende Anpassung
% %%%%%%%%%%%%%%

% \input{"IAB/latex/TeX-Folienformatierung_CD_2019"}
\input{"/Users/jonathanlatner/Google Drive/My Drive/IAB/latex/TeX-Folienformatierung_CD_2019"}

% Modify the section in toc template to enumerate
\setbeamertemplate{section in toc}{%
    \inserttocsectionnumber.~\inserttocsection\par
}

% use for subsections
% \setbeamertemplate{subsection in toc}{}
\setbeamertemplate{subsection in toc}{%
    \setlength{\parskip}{1mm}
        \hskip2mm -- \hskip1mm\inserttocsubsection\par
}


\usepackage{colortbl}
\definecolor{lightgray}{gray}{0.9}

\usepackage{listings} %include R code
\lstdefinestyle{mystyle}{
    backgroundcolor=\color{backcolour},   
    commentstyle=\color{codegreen},
    keywordstyle=\color{magenta},
    numberstyle=\tiny\color{codegray},
    stringstyle=\color{codepurple},
    basicstyle=\ttfamily\tiny,
    breakatwhitespace=false,         
    breaklines=true,                 
    captionpos=b,                    
    keepspaces=true,                 
    numbers=left,                    
    numbersep=5pt,                  
    showspaces=false,                
    showstringspaces=false,
    showtabs=false,                 
    columns=fullflexible,
    frame=single,
    tabsize=2
}
\lstset{style=mystyle}


\begin{document}


\frame[plain]{\titlepage}

\begin{spacing}{1.25}


\section{Introduction}\label{sec:intro}
\begin{frame}[c,plain]
\vskip-4mm
\begin{beamercolorbox}[wd=\boxwidth,ht=22.11mm]{transparent}%
    \vfill%
    \usebeamerfont{title}%
    \leftinsert%
    \MakeUppercase{Section \ref{sec:intro}: Introduction
} % <- Hier die Überschrift eintragen
\end{beamercolorbox}
\vskip-3mm
\pgfuseimage{rahmenlinie}
\end{frame}


%%%%%%%%%%%%%%%%%%%%%%%%%%%%%%%%%%%%%%%%
%%%%%%%%%%%%%%%%%%%%%%%%%%%%%%%%%%%%%%%%
\section{Simulated data}\label{sec:data}
%%%%%%%%%%%%%%%%%%%%%%%%%%%%%%%%%%%%%%%%
%%%%%%%%%%%%%%%%%%%%%%%%%%%%%%%%%%%%%%%%
\frame{\frametitle{Data}
From Reiter et al., 2014

``We use a simple simulation scenario that illustrates many of the main issues: protecting a $2^4$ binary table with fully synthetic data. For $i = 1, \dots, 1000 = n$, let $y_i = (y_{1i}, y_{2i}, y_{3i}, y_{4i})$ comprise four binary variables. Let each of the $K = 16$ possible combinations be denoted $c_k$, where $k = 1,\dots,16$.  Let $c_{16} = (0,0,0,0)$, and let $C_{-16} = (c_1,\dots,c_{15})$. We generate an observed dataset $D$ as follows.  For $i = 1,...,n-1 = 999$, sample $y_i$ from a multinomial distribution such that $p(y_i = c_k) = 1/15$ for all $c_k \in C-16$. Set $y_{1000} = c_{16}$. Since we do full synthesis, $X = \theta$''
}


\frame{\frametitle{Variable frequency}
\begin{figure}
    \caption{}
    \resizebox{\textwidth}{!}{\includegraphics{../../graphs/graph_numeric_frequency.pdf}}
    \label{fig:graph_numeric_frequency}
\end{figure}
}

\frame{\frametitle{Histogram}
\begin{figure}
    \caption{}
    \resizebox{\textwidth}{!}{\includegraphics{../../graphs/graph_numeric_histogram.pdf}}
    \label{fig:graph_numeric_histogram}
\end{figure}
}

\begin{frame}[fragile]
\frametitle{Synthpop}

\begin{lstlisting}
> sds <- syn(df_ods, m=1)
Warning: In your synthesis there are numeric variables with 5 or fewer levels: var1, var2, var3, var4.
Consider changing them to factors. You can do it using parameter 'minnumlevels'.

Synthesis
-----------
 var1 var2 var3 var4
\end{lstlisting}

notice the "Warning".  It means that the variables are being synthesized as numerical values  (0/1), and Synthpop is suggesting they should be synthesized as categorical values  (``0''/``1'')
\end{frame}

\frame{\frametitle{Compare frequency (numerical)}
\begin{figure}
    \caption{}
    \resizebox{\textwidth}{!}{\includegraphics{../../graphs/graph_numeric_compare_frequency.pdf}}
    \label{fig:graph_numeric_compare_frequency}
\end{figure}
}

\frame{\frametitle{Compare histogram (numerical)}
\begin{figure}
    \caption{}
    \resizebox{\textwidth}{!}{\includegraphics{../../graphs/graph_numeric_compare_histogram.pdf}}
    \label{fig:graph_numeric_compare_histogram}
\end{figure}
}

\frame{\frametitle{Compare histogram (numerical) x 100 synthetic datasets}
\begin{figure}
    \caption{}
    \resizebox{\textwidth}{!}{\includegraphics{../../graphs/graph_numeric_compare_histogram_100.pdf}}
    \label{fig:graph_numeric_compare_histogram_100}
\end{figure}
}

\frame{\frametitle{Compare frequency (categorical)}
\begin{figure}
    \caption{}
    \resizebox{\textwidth}{!}{\includegraphics{../../graphs/graph_categorical_compare_frequency.pdf}}
    \label{fig:graph_categorical_compare_frequency}
\end{figure}
}

\frame{\frametitle{Compare histogram (categorical)}
\begin{figure}
    \caption{}
    \resizebox{\textwidth}{!}{\includegraphics{../../graphs/graph_categorical_compare_histogram.pdf}}
    \label{fig:graph_categorical_compare_histogram}
\end{figure}
}

\frame{\frametitle{Compare histogram (categorical) x 100 synthetic datasets}
\begin{figure}
    \caption{}
    \resizebox{\textwidth}{!}{\includegraphics{../../graphs/graph_categorical_compare_histogram_100.pdf}}
    \label{fig:graph_categorical_compare_histogram_100}
\end{figure}
}

%%%%%%%%%%%%%%%%%%%%%%%%%%%%%%%%%%%%%%%%
%%%%%%%%%%%%%%%%%%%%%%%%%%%%%%%%%%%%%%%%
\section{Measuring utility and privacy}\label{sec:measuring}
%%%%%%%%%%%%%%%%%%%%%%%%%%%%%%%%%%%%%%%%
%%%%%%%%%%%%%%%%%%%%%%%%%%%%%%%%%%%%%%%%
\frame{\frametitle{Comparing utility measures}
\begin{figure}
    \caption{Utility measures close to 0, i.e. high utility}
    \resizebox{\textwidth}{!}{\includegraphics{../../graphs/graph_compare_utility.pdf}}
    \label{fig:graph_compare_utility}
\end{figure}
}

\frame{\frametitle{Comparing privacy measures}
all privacy measures close to 0, i.e. low privacy risk
}

%%%%%%%%%%%%%%%%%%%%%%%%%%%%%%%%%%%%%%%%
%%%%%%%%%%%%%%%%%%%%%%%%%%%%%%%%%%%%%%%%
\section{How do we explain this?}\label{sec:explain}
%%%%%%%%%%%%%%%%%%%%%%%%%%%%%%%%%%%%%%%%
%%%%%%%%%%%%%%%%%%%%%%%%%%%%%%%%%%%%%%%%
\frame{\frametitle{How do we explain this?}
\begin{figure}
    \caption{}
    \resizebox{\textwidth}{!}{\includegraphics{../../graphs/graph_tree_combined.pdf}}
    \label{fig:graph_tree_combined}
\end{figure}
}

\frame{\frametitle{Histogram with differential privacy x 100 simulations}
\begin{figure}
    \caption{}
    \resizebox{\textwidth}{!}{\includegraphics{../../graphs/graph_dp_compare_histogram_100.pdf}}
    \label{fig:graph_dp_compare_histogram_100}
\end{figure}
}



\end{spacing}
\end{document}

