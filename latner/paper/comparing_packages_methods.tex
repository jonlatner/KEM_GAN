
\documentclass[10pt]{article} % For LaTeX2e
\usepackage[preprint]{tmlr}
% If accepted, instead use the following line for the camera-ready submission:
%\usepackage[accepted]{tmlr}
% To de-anonymize and remove mentions to TMLR (for example for posting to preprint servers), instead use the following:
%\usepackage[preprint]{tmlr}

% Optional math commands from https://github.com/goodfeli/dlbook_notation.
\input{math_commands.tex}

\usepackage{hyperref}
\usepackage{url}


\title{Comparing efficiency, utility, and privacy between synthetic data packages and methods}

% Authors must not appear in the submitted version. They should be hidden
% as long as the tmlr package is used without the [accepted] or [preprint] options.
% Non-anonymous submissions will be rejected without review.

\author{\name Jörg Drechsler \email joerg.drechsler@iab.de \\
      \addr Institute for Employment Research\\
      Nuremberg, Germany
      \AND
      \name Jonathan Latner \email jonathan.latner@iab.de \\
      \addr Institute for Employment Research\\
      Nuremberg, Germany
      \AND
      \name Marcel Neunhoeffer \email marcel.neunhoeffer@iab.de\\
      \addr Institute for Employment Research\\
      Nuremberg, Germany}

% The \author macro works with any number of authors. Use \AND 
% to separate the names and addresses of multiple authors.

\newcommand{\fix}{\marginpar{FIX}}
\newcommand{\new}{\marginpar{NEW}}

%\def\month{MM}  % Insert correct month for camera-ready version
%\def\year{YYYY} % Insert correct year for camera-ready version
\def\openreview{\url{https://openreview.net/forum?id=XXXX}} % Insert correct link to OpenReview for camera-ready version


\begin{document}


\maketitle
\today
\tableofcontents
\begin{abstract}

something here

\end{abstract}

\section{Background}

The origins of releasing synthetic data instead of actual data are often understood to begin with Rubin \citeyearpar{rubin1993statistical}, who proposed a method for fully synthetic data, and Little \citeyearpar{little1993statistical}, who proposed a method for partially synthetic data.  Here, when we refer to data, we are referring to microdata (i.e. one observation per individual unit) as opposed to tabular data (i.e. summary data).  There are multiple reasons why releasing synthetic data are preferable to releasing actual data.  Privacy protection is a primary reason.  Theoretically, releasing synthetic data means altering actual data in some way so that the released data do not contain information on a real individual unit (person, firm, etc.).  The advantage is more users can use data that would otherwise be not easily available.  The disadvantage is that the higher the level of privacy protection the lower the level of utility of the data or the degree to which the real data match the synthetic data.  This is known as the risk-accuracy trade-off \citeyearpar{reiter2010releasing}.  While a few recent papers compare and contrast statistical packages the create synthetic data on the trade-off between privacy and utility \citep{little2022comparing,dankar2021fake}, there is little discussion of the computational power or `efficiency' required to create synthetic data.  In this paper, we contribute to the literature by adding efficiency as a third dimension to the relationship between privacy and utility.

This paper seeks to explore the relationship between computational efficiency (duration in time) and measures for utility/privacy between different packages that create synthetic data and the methods they use.  The starting point for this paper are as follows:

\begin{itemize}
    \item Working paper by Little et al., 2022
    \item Synthpop is the `winner' among datasynthesizers
    \begin{itemize}
        \item Highest level of utility
        \item Highest level of privacy
        \item Highest level of efficiency
    \end{itemize}
    \item How is Synthpop so good?
    \item What is the difference between CART and Synthpop?
\end{itemize}

\section{Goals}

The key points this paper seeks to make are as follows:

\begin{itemize}
    \item Synthpop 
    \begin{itemize}
        \item Performs well in low-dimensional data
        \item Note: Its not clear how exactly to define dimensionality
    \end{itemize}
    \item CTGAN
    \begin{itemize}
        \item Efficiency
        \begin{itemize}
            \item In high-dimensional data, CTGAN can be much faster than Synthpop
            \item One reason CTGAN/GANs are slow is epochs
            \item We may not need so many epochs to synthesize single table, maybe as low as 10
        \end{itemize}
        \item Utility
        \begin{itemize}
            \item Regardless of dimensionality, utility is still low in CTGAN
            \item Can we build a better GAN?
        \end{itemize}
    \end{itemize}
    \item Datasynthesizer
    \begin{itemize}
        \item In low-dimensional data, performs similar to Synthpop
        \item In high-dimensional data, performs better than Synthpop/CTGAN
    \end{itemize}
    \item Method vs. Package
    \begin{itemize}
        \item Synthpop is not CART and CART is not Synthpop
        \item CTGAN is not the only GAN
    \end{itemize}
    \item Privacy
    \begin{itemize}
        \item How do we measure privacy in a universal way?
        \item What is privacy?
    \end{itemize}
\end{itemize}

\section{Preliminaries}
\label{sec:preliminaries}

\subsection{Efficiency}

\subsection{Utility}

\subsection{Privacy}


\subsubsection*{Acknowledgments}
Use unnumbered third level headings for the acknowledgments. All
acknowledgments, including those to funding agencies, go at the end of the paper.
Only add this information once your submission is accepted and deanonymized. \citep{dankar2021fake}

\bibliography{references}
\bibliographystyle{tmlr}

\appendix
\section{Appendix}
You may include other additional sections here.

\end{document}
