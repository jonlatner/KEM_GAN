% \input{"IAB/latex/TeX-Folienformat.tex"}
\input{"/Users/jonathanlatner/Google Drive/My Drive/IAB/latex/TeX-Folienformat.tex"}

\documentclass[t,8pt,utfx8]{beamer}
\usepackage{booktabs}
\usepackage{setspace}
\usepackage{parskip}
\usepackage{graphicx}
\usepackage{subcaption}
\setbeamertemplate{caption}[numbered]
\newcommand{\sprache}{\englisch}
\renewcommand{\thesubsection}{\alph{subsection})}
\usepackage[cal=pxtx, scr=dutchcal]{mathalpha}
\usepackage{forest}



\definecolor{codegreen}{rgb}{0,0.6,0}
\definecolor{codegray}{rgb}{0.5,0.5,0.5}
\definecolor{codepurple}{rgb}{0.58,0,0.82}
\definecolor{backcolour}{rgb}{0.95,0.95,0.92}


\usepackage{listings}

% Define style for R code
\lstset{
  language=R,
  basicstyle=\ttfamily\small,
  keywordstyle=\color{blue},
  stringstyle=\color{red},
  commentstyle=\color{green},
  showstringspaces=false,
  numbers=left,
  numberstyle=\tiny\color{gray},
  stepnumber=1,
  numbersep=5pt,
  breaklines=true,
  frame=single
}

\newcommand{\btVFill}{\vskip0pt plus 1filll}


\title{Buyer Beware: Understanding the trade-off between utility and risk in CART based models using simulation data}

\author{Jonathan Latner, PhD \newline Dr. Marcel Neunhoeffer \newline Prof. Dr. Jörg Drechsler}

\newcounter{noauthorlines}
\setcounter{noauthorlines}{2} % Wert für 2 Autoren über 2 Zeilen. Ggf. anpassen

% %%%%%%%%%%%%%%
% Ende Anpassung
% %%%%%%%%%%%%%%

% \input{"IAB/latex/TeX-Folienformatierung_CD_2019"}
\input{"/Users/jonathanlatner/Google Drive/My Drive/IAB/latex/TeX-Folienformatierung_CD_2019"}

% Modify the section in toc template to enumerate
\setbeamertemplate{section in toc}{%
    \inserttocsectionnumber.~\inserttocsection\par
}

% use for subsections
% \setbeamertemplate{subsection in toc}{}
\setbeamertemplate{subsection in toc}{%
    \setlength{\parskip}{1mm}
        \hskip2mm -- \hskip1mm\inserttocsubsection\par
}


\usepackage{colortbl}
\definecolor{lightgray}{gray}{0.9}

\usepackage{listings} %include R code
\lstdefinestyle{mystyle}{
    backgroundcolor=\color{backcolour},   
    commentstyle=\color{codegreen},
    keywordstyle=\color{magenta},
    numberstyle=\tiny\color{codegray},
    stringstyle=\color{codepurple},
    basicstyle=\ttfamily\tiny,
    breakatwhitespace=false,         
    breaklines=true,                 
    captionpos=b,                    
    keepspaces=true,                 
    numbers=left,                    
    numbersep=5pt,                  
    showspaces=false,                
    showstringspaces=false,
    showtabs=false,                 
    columns=fullflexible,
    frame=single,
    tabsize=2
}
\lstset{style=mystyle}


\begin{document}


\frame[plain]{\titlepage}

\begin{spacing}{1.25}

%%%%%%%%%%%%%%%%%%%%%%%%%%%%%%%%%%%%%%%%
%%%%%%%%%%%%%%%%%%%%%%%%%%%%%%%%%%%%%%%%
\section{The data}\label{sec:data}
%%%%%%%%%%%%%%%%%%%%%%%%%%%%%%%%%%%%%%%%
%%%%%%%%%%%%%%%%%%%%%%%%%%%%%%%%%%%%%%%%
\begin{frame}[c,plain]
\vskip-4mm
\begin{beamercolorbox}[wd=\boxwidth,ht=22.11mm]{transparent}%
    \vfill%
    \usebeamerfont{title}%
    \leftinsert%
    \MakeUppercase{Section \ref{sec:data}: Generate the original and synthetic data
} % <- Hier die Überschrift eintragen
\end{beamercolorbox}
\vskip-3mm
\pgfuseimage{rahmenlinie}
\begin{itemize}
    \item Borrowing from Reiter et al. (2014), we create a data set with $n=1000$ and 4 dichotomous, categorical variables. 
    \item The first 999 observations to be a random sample from a multinomial distribution for all combinations of $var1(0,1), var2(0,1), var3(0,1), var4(0,1)$ except the last one
    \item The last (1000$^{th}$) observation is ($var1=1,var2=1,var3=1,var4=1$). 
\end{itemize}

\end{frame}

\frame{\frametitle{Generate original data using a simulation}
\begin{minipage}{0.48\textwidth}
    \begin{figure}
        \centering
        \caption{Frequency}
        \includegraphics[width=\textwidth]{../../graphs/graph_cart_frequency.pdf}
    \end{figure}
\end{minipage}
\hfill
\begin{minipage}{0.48\textwidth}
    \begin{figure}
        \centering
        \caption{Histogram}
        \includegraphics[width=\textwidth]{../../graphs/graph_cart_histogram.pdf}
    \end{figure}
\end{minipage}

}


\frame{\frametitle{Generate synthetic data with CART (synthpop)}

\begin{minipage}{0.48\textwidth}
    \begin{figure}
        \centering
        \caption{Frequency}
        \includegraphics[width=\textwidth]{../../graphs/graph_cart_frequency_compare.pdf}
    \end{figure}
\end{minipage}
\hfill
\begin{minipage}{0.48\textwidth}
    \begin{figure}
        \centering
        \caption{Histogram}
        \includegraphics[width=\textwidth]{../../graphs/graph_cart_histogram_compare.pdf}
    \end{figure}
\end{minipage}

}


\frame{\frametitle{Compare histogram x 10 synthetic datasets}
\begin{figure}
    \caption{Multiple synthetic data sets does not reduce privacy risk}
    \resizebox{\textwidth}{!}{\includegraphics{../../graphs/graph_cart_histogram_compare_10.pdf}}
    \label{fig:graph_cart_histogram_compare_10}
\end{figure}
}

\frame{\frametitle{Summary}
\begin{itemize}
    \item The problem (in our data): Synthetic data from CART models are disclosive
    \item The reason: 
    \begin{itemize}
        \item A record can only be in the synthetic data if it is also in the original data (in this simulated data).   
        \item Or the opposite: if a record is not in the original data, then it can never be in the synthetic data.
    \end{itemize}  
    \item Next section: Can an attacker identify the disclosure?
\end{itemize}
}

%%%%%%%%%%%%%%%%%%%%%%%%%%%%%%%%%%%%%%%%
%%%%%%%%%%%%%%%%%%%%%%%%%%%%%%%%%%%%%%%%
\section{The attack}\label{sec:attack}
%%%%%%%%%%%%%%%%%%%%%%%%%%%%%%%%%%%%%%%%
%%%%%%%%%%%%%%%%%%%%%%%%%%%%%%%%%%%%%%%%
\begin{frame}[c,plain]
\vskip-4mm
\begin{beamercolorbox}[wd=\boxwidth,ht=22.11mm]{transparent}%
    \vfill%
    \usebeamerfont{title}%
    \leftinsert%
    \MakeUppercase{Section \ref{sec:attack}: The attack
} % <- Hier die Überschrift eintragen
\end{beamercolorbox}
\vskip-3mm
\pgfuseimage{rahmenlinie}

\end{frame}

\frame{\frametitle{Describing the attack}
\begin{itemize}
    \item We assume a `strong' attacker similar to the attack model in differential privacy (DP). 
    \item An attacker has the following knowledge
    \begin{itemize}
        \item Knows the SDG model type (i.e. sequential CART).
        \item Knowledge of all observations in the data except the last one.  
        \item The 16 possible combinations that the last one could be.
    \end{itemize}
    \item The attacker sees the synthetic data
    \item The attacker runs the same synthetic data model (SDG) for all of the 16 different possibilities.  
    \item Then they update their beliefs about what the last record could be
\end{itemize}
}


\frame{\frametitle{Illustrating the attack with CART (default parameters)}
\vskip -3mm
\begin{figure}
    \caption{Histogram of 16 worlds x 100 synthetic datasets}
    \vskip -4mm
    \resizebox{\textwidth}{!}{\includegraphics{../../graphs/graph_attacker_default.pdf}}
    \label{fig:graph_attacker_default}
\end{figure}
}


\frame{\frametitle{Summary}
\begin{itemize}
    \item In our attack with our assumptions, the attacker can easily identify the last record
    \item The reason (to repeat): 
    \begin{itemize}
        \item A record can only be in the synthetic data if it is also in the original data (in this simulated data).   
        \item Or the opposite: if a record is not in the original data, then it can never be in the synthetic data.
    \end{itemize}  
    \item Next section: Can we measure this disclosure?
\end{itemize}

}


%%%%%%%%%%%%%%%%%%%%%%%%%%%%%%%%%%%%%%%%
%%%%%%%%%%%%%%%%%%%%%%%%%%%%%%%%%%%%%%%%
\section{Measuring privacy}\label{sec:privacy}
%%%%%%%%%%%%%%%%%%%%%%%%%%%%%%%%%%%%%%%%
%%%%%%%%%%%%%%%%%%%%%%%%%%%%%%%%%%%%%%%%
\begin{frame}[c,plain]
\vskip-4mm
\begin{beamercolorbox}[wd=\boxwidth,ht=22.11mm]{transparent}%
    \vfill%
    \usebeamerfont{title}%
    \leftinsert%
    \MakeUppercase{Section \ref{sec:privacy}: Measuring privacy
} % <- Hier die Überschrift eintragen
\end{beamercolorbox}
\vskip-3mm
\pgfuseimage{rahmenlinie}

\end{frame}


\frame{\frametitle{Common privacy measures - Synthpop (Raab et al., 2024)}
\begin{itemize}
    \item Replicated uniques (\#)
    \item Identity disclosure (\%): the ability to identify individuals in the data from a set of known characteristics or `keys' ($q$).  
    \item Attribute disclosure (\%): the ability to find out from the keys something, not previously known or `target' ($t$)
\end{itemize}
}

\frame{\frametitle{Attribute disclosure}
\small
% \% Disclosive in Original: is the percent of records in original where the keys ($q$) identify a unique target ($t$).  In our case, this is always 0 because there are many (1110) and 1 unique record (1111).

% \begin{equation}
% D_{orig} = 100 \sum^{q} \sum^{t} (d_{tq} \mid pd_{tq} = 1) / N_{d}
% \end{equation}

Disclosive in Synthetic: is the percent of records in SD where the keys ($q$) identify a unique target ($t$).  In our case, when there is no unique record in the SD, this equals the percent of records with 1110 in SD.

\begin{equation}
D_{syn} = 100 \sum^{q} \sum^{t} (s_{tq} \mid ps_{tq} = 1) / N_{s}
\end{equation}

Disclosive in Synthetic: is the percent of all records in OD where  $q$ in SD is disclosive (i.e.,  $t$  values for  $q$  are constant in SD).  In our case, this is the percent records with 1111 or 1110 in the OD (i.e. 67\%) when there is no unique record in SD.

\begin{equation}
DiS = 100 \sum^{q} \sum^{i=1,\dots,T} \sum^{j=1,\dots,T} (d_{iq} \mid ps_{jq} = 1) / N_{d}
\end{equation}

\% Disclosive in Synthetic Correct in Original: percent of all records in OD where  $q$ in SD is disclosive and the disclosed  $t$  value matches the true  $t$  value in OD.  In our case, this is the percent records with 1110 in the OD (i.e. 66\%) when there is no unique record in SD.

\begin{equation}
DiSCO = 100 \sum^{q} \sum^{t} (d_{tq} \mid ps_{tq} = 1) / N_{s}
\end{equation}

}

\begin{frame}[fragile]
\frametitle{Comparing disclosure risk measures}

\begin{minipage}[t]{0.48\textwidth}
    \begin{table}[]
        \centering
        \caption{x 1 synthetic data set (seed = 1237)}
        % latex table generated in R 4.5.0 by xtable 1.8-4 package
% Thu Dec  4 16:03:52 2025
\begin{tabular}{lrrr}
  \toprule
Data & Unique & Identity Risk ($repU$) & Attribute Risk ($DiSCO$) \\ 
  \midrule
Original &   1 & 0.00 & 0.00 \\ 
  Synthetic &   1 & 0.00 & 0.00 \\ 
   \bottomrule
\end{tabular}

        \label{table:disclosure_risk_1}
    \end{table}
\end{minipage}%
\hfill%
\begin{minipage}[t]{0.48\textwidth}
    \begin{table}[]
        \centering
        \caption{x 10 synthetic data sets}
        % latex table generated in R 4.4.0 by xtable 1.8-4 package
% Thu Dec 19 16:23:49 2024
\begin{tabular}{lrrr}
  \toprule
data & identity & unique & attribute \\ 
  \midrule
Original & 0.00 & 1 & 0.00 \\ 
  Synthetic & 0.00 & see table \ref{table:frequency_10_data_sets} & 1.32 \\ 
   \bottomrule
\end{tabular}

        \label{table:disclosure_risk_10}
    \end{table}
\end{minipage}
\end{frame}


\begin{frame}[fragile]
\frametitle{Underlying information}

\begin{minipage}[t]{0.48\textwidth}
    \begin{table}[]
        \centering
        \caption{Frequency statistics}
        \rowcolors{1}{white}{lightgray}
        \resizebox{.9\textwidth}{!}{% latex table generated in R 4.4.0 by xtable 1.8-4 package
% Mon Dec 16 16:27:40 2024
\begin{tabular}{lrrrrrrrrrrr}
  \toprule
   & \multicolumn{1}{l}{Original} & \multicolumn{10}{c}{Synthetic Data} \\ \cmidrule(lr){3-12}
 Combine & 0 & 1 & 2 & 3 & 4 & 5 & 6 & 7 & 8 & 9 & 10 \\ 
 \midrule
0000 & 69 & 68 & 66 & 71 & 73 & 76 & 62 & 72 & 52 & 64 & 67 \\ 
  0001 & 60 & 60 & 53 & 57 & 56 & 58 & 60 & 67 & 67 & 57 & 67 \\ 
  0010 & 79 & 77 & 71 & 73 & 71 & 71 & 84 & 65 & 70 & 77 & 74 \\ 
  0011 & 62 & 70 & 51 & 56 & 68 & 63 & 55 & 74 & 57 & 68 & 52 \\ 
  0100 & 68 & 73 & 63 & 80 & 54 & 61 & 79 & 65 & 73 & 66 & 71 \\ 
  0101 & 64 & 60 & 77 & 49 & 66 & 52 & 90 & 52 & 53 & 65 & 71 \\ 
  0110 & 60 & 58 & 68 & 66 & 61 & 69 & 56 & 67 & 65 & 64 & 53 \\ 
  0111 & 75 & 72 & 91 & 86 & 81 & 80 & 77 & 82 & 77 & 75 & 72 \\ 
  1000 & 74 & 77 & 84 & 80 & 73 & 70 & 81 & 82 & 65 & 76 & 73 \\ 
  1001 & 63 & 69 & 66 & 57 & 68 & 73 & 56 & 68 & 75 & 78 & 55 \\ 
  1010 & 54 & 50 & 54 & 57 & 51 & 47 & 50 & 39 & 62 & 58 & 54 \\ 
  1011 & 69 & 61 & 59 & 77 & 71 & 66 & 69 & 75 & 69 & 68 & 81 \\ 
  1100 & 77 & 79 & 77 & 76 & 83 & 78 & 66 & 65 & 88 & 70 & 89 \\ 
  1101 & 59 & 65 & 52 & 54 & 57 & 66 & 67 & 59 & 65 & 49 & 60 \\ 
  1110 & 66 & 60 & 68 & 60 & 64 & 68 & 47 & 65 & 62 & 64 & 60 \\ 
  1111 & 1 & 1 & 0 & 1 & 3 & 2 & 1 & 3 & 0 & 1 & 1 \\ 
   \bottomrule
\end{tabular}
}
        % % latex table generated in R 4.4.0 by xtable 1.8-4 package
% Mon Dec 16 16:27:40 2024
\begin{tabular}{lrrrrrrrrrrr}
  \toprule
   & \multicolumn{1}{l}{Original} & \multicolumn{10}{c}{Synthetic Data} \\ \cmidrule(lr){3-12}
 Combine & 0 & 1 & 2 & 3 & 4 & 5 & 6 & 7 & 8 & 9 & 10 \\ 
 \midrule
0000 & 69 & 68 & 66 & 71 & 73 & 76 & 62 & 72 & 52 & 64 & 67 \\ 
  0001 & 60 & 60 & 53 & 57 & 56 & 58 & 60 & 67 & 67 & 57 & 67 \\ 
  0010 & 79 & 77 & 71 & 73 & 71 & 71 & 84 & 65 & 70 & 77 & 74 \\ 
  0011 & 62 & 70 & 51 & 56 & 68 & 63 & 55 & 74 & 57 & 68 & 52 \\ 
  0100 & 68 & 73 & 63 & 80 & 54 & 61 & 79 & 65 & 73 & 66 & 71 \\ 
  0101 & 64 & 60 & 77 & 49 & 66 & 52 & 90 & 52 & 53 & 65 & 71 \\ 
  0110 & 60 & 58 & 68 & 66 & 61 & 69 & 56 & 67 & 65 & 64 & 53 \\ 
  0111 & 75 & 72 & 91 & 86 & 81 & 80 & 77 & 82 & 77 & 75 & 72 \\ 
  1000 & 74 & 77 & 84 & 80 & 73 & 70 & 81 & 82 & 65 & 76 & 73 \\ 
  1001 & 63 & 69 & 66 & 57 & 68 & 73 & 56 & 68 & 75 & 78 & 55 \\ 
  1010 & 54 & 50 & 54 & 57 & 51 & 47 & 50 & 39 & 62 & 58 & 54 \\ 
  1011 & 69 & 61 & 59 & 77 & 71 & 66 & 69 & 75 & 69 & 68 & 81 \\ 
  1100 & 77 & 79 & 77 & 76 & 83 & 78 & 66 & 65 & 88 & 70 & 89 \\ 
  1101 & 59 & 65 & 52 & 54 & 57 & 66 & 67 & 59 & 65 & 49 & 60 \\ 
  1110 & 66 & 60 & 68 & 60 & 64 & 68 & 47 & 65 & 62 & 64 & 60 \\ 
  1111 & 1 & 1 & 0 & 1 & 3 & 2 & 1 & 3 & 0 & 1 & 1 \\ 
   \bottomrule
\end{tabular}

        \label{table:frequency_10_data_sets}
    \end{table}
\end{minipage}%
\hfill%
\begin{minipage}[t]{0.48\textwidth}
    \begin{table}[h!]
        \centering
        \caption{Attribute risk measures from 10 synthetic data sets}
        \rowcolors{1}{white}{lightgray}
        % latex table generated in R 4.4.0 by xtable 1.8-4 package
% Thu Dec 19 16:23:49 2024
\begin{tabular}{lrrrr}
  \toprule
m & Dsyn & iS & DiS & DiSCO \\ 
  \midrule
1 & 0 & 100 & 0 & 0 \\ 
  2 & 6.8 & 100 & 6.7 & 6.6 \\ 
  3 & 0 & 100 & 0 & 0 \\ 
  4 & 0 & 100 & 0 & 0 \\ 
  5 & 0 & 100 & 0 & 0 \\ 
  6 & 0 & 100 & 0 & 0 \\ 
  7 & 0 & 100 & 0 & 0 \\ 
  8 & 6.2 & 100 & 6.7 & 6.6 \\ 
  9 & 0 & 100 & 0 & 0 \\ 
  10 & 0 & 100 & 0 & 0 \\ 
  Average & 1.3 & 100 & 1.34 & 1.32 \\ 
   \bottomrule
\end{tabular}

        \label{table:attribute_risk_10}
    \end{table}
\end{minipage}


\end{frame}

\frame{\frametitle{Summary}
\begin{itemize}
    \item Using common privacy measures, CART generates synthetic data with low risk
    \item 1 measure indicates there may be a problem, but all the other measures indicate there is no problem.
    \item However (and this is the point):
    \begin{itemize}
         \item We know there is a problem (because we created it)
         \item We know that common measures do not capture the problem 
    \end{itemize} 
    \item We are also not alone in identifying this problem (Manrique-Vallier and Hu, 2018)
\end{itemize}
}


%%%%%%%%%%%%%%%%%%%%%%%%%%%%%%%%%%%%%%%%
%%%%%%%%%%%%%%%%%%%%%%%%%%%%%%%%%%%%%%%%
\section{Solution}\label{sec:solution}
%%%%%%%%%%%%%%%%%%%%%%%%%%%%%%%%%%%%%%%%
%%%%%%%%%%%%%%%%%%%%%%%%%%%%%%%%%%%%%%%%
\begin{frame}[c,plain]
\vskip-4mm
\begin{beamercolorbox}[wd=\boxwidth,ht=22.11mm]{transparent}%
    \vfill%
    \usebeamerfont{title}%
    \leftinsert%
    \MakeUppercase{Section \ref{sec:solution}: Solution
} % <- Hier die Überschrift eintragen
\end{beamercolorbox}
\vskip-3mm
\pgfuseimage{rahmenlinie}
\end{frame}

\frame{\frametitle{The good news: solutions}
\begin{itemize}
    \item Reduce utility by preventing overfitting
    \begin{itemize}
        \item minbucket = 75 (default = 5): increase the minimum number of observations in any terminal node
        \item complexity parameter (cp) = 0.05 (default = 1e$^{-8}$): decrease the size of the tree
        \item Other options also exist
        \begin{itemize}
            \item Comparison to differential privacy ($\epsilon$-DP)
        \end{itemize}
    \end{itemize}
\end{itemize}
}

\frame{\frametitle{Generate synthetic data with CART (modified parameters)}
\begin{minipage}{0.48\textwidth}
    \begin{figure}
        \centering
        \caption{minbucket}
        \includegraphics[width=\textwidth]{../../graphs/graph_cart_modified_mb_histogram_compare_10_v2.pdf}
    \end{figure}
\end{minipage}
\hfill
\begin{minipage}{0.48\textwidth}
    \begin{figure}
        \centering
        \caption{cp}
        \includegraphics[width=\textwidth]{../../graphs/graph_cart_modified_cp_histogram_compare_10_v2.pdf}
    \end{figure}
\end{minipage}
}

\frame{\frametitle{Other options: generate noise with $\epsilon$-DP}
\begin{figure}[!h]
    \centering
    \caption{Datasynthesizer with DP}
    \resizebox{\textwidth}{!}{\includegraphics{../../graphs/graph_dp_datasynthesizer_compare_histogram_10_v2.pdf}}
    \label{fig:dp_datasynthesizer}
\end{figure}
}


\frame{\frametitle{Illustrating the attack with CART (modified parameters)}
\vskip -4mm
\begin{minipage}{0.48\textwidth}
    \begin{figure}
        \centering
        \caption{mb = 75}
        \includegraphics[width=\textwidth]{../../graphs/graph_attacker_modified_mb_v2.pdf}
    \end{figure}
\end{minipage}
\hfill
\begin{minipage}{0.48\textwidth}
    \begin{figure}
        \centering
        \caption{cp = 0.05}
        \includegraphics[width=\textwidth]{../../graphs/graph_attacker_modified_cp_v2.pdf}
    \end{figure}
\end{minipage}
}

\frame{\frametitle{The bad news}
\begin{itemize}
    \item We don't know how to identify the privacy risk
    \item We have to know a problem exists before we would do something about it
\end{itemize}
}

%%%%%%%%%%%%%%%%%%%%%%%%%%%%%%%%%%%%%%%%
%%%%%%%%%%%%%%%%%%%%%%%%%%%%%%%%%%%%%%%%
\section{Conclusion}\label{sec:conclusion}
%%%%%%%%%%%%%%%%%%%%%%%%%%%%%%%%%%%%%%%%
%%%%%%%%%%%%%%%%%%%%%%%%%%%%%%%%%%%%%%%%
\begin{frame}[c,plain]
\vskip-4mm
\begin{beamercolorbox}[wd=\boxwidth,ht=22.11mm]{transparent}%
    \vfill%
    \usebeamerfont{title}%
    \leftinsert%
    \MakeUppercase{Section \ref{sec:conclusion}: Conclusion} % <- Hier die Überschrift eintragen
\end{beamercolorbox}
\vskip-3mm
\pgfuseimage{rahmenlinie}
\end{frame}

\frame{\frametitle{Summary}
\begin{itemize}
    \item It has long been understood that there is a trade-off between utility and risk
    \item Previous research indicated that CART models were less sensitive to this trade-off than other SDGs 
    \item Using a simulated data set, we show that CART are sensitive to this trade-off
    \item The good news: It is possible to reduce risk in CART with parameters
    \item The bad news:
    \begin{itemize}
        \item Common privacy metrics do not capture risk in our simulated data
        \item We must sacrifice utility
    \end{itemize}
    \item Question: If you did not know there was a problem, why would you sacrifice utility?

\end{itemize}
}

\frame{\frametitle{Thank you}

Jonathan Latner: \url{jonathan.latner@iab.de} \\

Reproducible code: \url{https://github.com/jonlatner/KEM\_GAN/tree/main/latner/projects/simulation} 

}



\end{spacing}
\end{document}

